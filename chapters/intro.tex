% !TeX root = ../main.tex

\chapter{引言}
商品拜物教是马克思在《资本论》中提出的概念,用以揭示资本主义市场中商品所呈现的神秘性质及其对人们意识的影响\cite{dekaer.makesi2012zibenlun}。马克思指出,商品乍看似乎很平凡,但分析表明它“充满形而上学的微妙和神学的怪诞”。这种“怪诞”在于人们仿佛相信商品本身拥有价值和力量,因而崇拜物品而忽视创造物品的人的劳动和社会关系\cite{song2024guan}。这一理论为我们理解当今消费社会中的诸多现象提供了重要视角。

近年来在潮流手办(潮玩)市场上出现的限量发售和二级溢价交易热潮,就是商品拜物教在当代的一种表现\cite{WJXB202502001,XJZS202406003}。本文以限量潮流手办的生产—流通—消费链为例,在马克思商品拜物教理论框架下,对潮流手办市场的运行机制和问题进行分析。文章将首先概述潮流手办产业及商品生命周期,然后从拜物教视角揭示潮玩交易中劳动、价值和社会关系被遮蔽的现象,并结合实际交易数据予以说明。继而讨论商品拜物教如何服务于资本获取剩余价值的机制,最后提出基于当代中国马克思主义的批判与调控建议。